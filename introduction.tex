\section{Introduction}\label{sec:introduction}

\toolname is a set of tools designed for understanding and testing
of Java programs. The main advantage of \toolname is that it does
not require the Java source code to perform its activities. Such a
characteristic allows, for instance, to use the tool for testing
Java-based components.

This report describes the functionalities of \toolname\ --
\version and how the tester can use it to create testing projects
through its graphical interface. Although \toolname can be run
using scripts, in this report this aspect is not addressed. To
illustrate the operational aspects of \toolname, through its
graphical interface, we are using a simple example, adapted from
Orso \etal~\cite{Orso01UCMS}, that simulates the behavior of a
vending machine.

The rest of this report is organized as follows.
Section~\ref{sec:background} presents some background information
about Java bytecode and a detailed description about the
underlying models used by \toolname to derive intra-method testing
requirements. A brief description of program slicing and
complexity metrics are also presented in
Section~\ref{sec:background}. Section~\ref{sec:example} describes
the example that we will use to illustrate the functionalities of
\toolname. Section~\ref{sec:project} describes how to create a
testing project. Section~\ref{sec:coverage} illustrates how to use
\toolname as a coverage analysis testing tool for Java
programs/components. Section~\ref{sec:slice} shows how to use
\toolname functionalities to localize faults. In
Section~\ref{sec:metrics} describes the set of static metrics
implemented in \toolname. Finally, in Section~\ref{sec:evolution},
we present the perspectives for \toolname evolution.
